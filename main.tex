\documentclass{ctexart}

\usepackage{amsmath, amsthm, amssymb, amsfonts}
\usepackage{thmtools}
\usepackage{graphicx}
\usepackage{setspace}
\usepackage{geometry}
\usepackage{float}
\usepackage{hyperref}
\usepackage[utf8]{inputenc}
\usepackage[english]{babel}
\usepackage{framed}
\usepackage[dvipsnames]{xcolor}
\usepackage{tcolorbox}

\colorlet{LightGray}{White!90!Periwinkle}
\colorlet{LightOrange}{Orange!15}
\colorlet{LightGreen}{Green!15}

\newcommand{\HRule}[1]{\rule{\linewidth}{#1}}

\declaretheoremstyle[name=Theorem,]{thmsty}
\declaretheorem[style=thmsty,numberwithin=section]{theorem}
\tcolorboxenvironment{theorem}{colback=LightGray}

\declaretheoremstyle[name=Proposition,]{prosty}
\declaretheorem[style=prosty,numberlike=theorem]{proposition}
\tcolorboxenvironment{proposition}{colback=LightOrange}

\declaretheoremstyle[name=Principle,]{prcpsty}
\declaretheorem[style=prcpsty,numberlike=theorem]{principle}
\tcolorboxenvironment{principle}{colback=LightGreen}

\setstretch{1.2}
\geometry{
    textheight=9in,
    textwidth=5.5in,
    top=1in,
    headheight=12pt,
    headsep=25pt,
    footskip=30pt
}

% ------------------------------------------------------------------------------

\begin{document}

% ------------------------------------------------------------------------------
% Cover Page and ToC
% ------------------------------------------------------------------------------

\title{ \normalsize \textsc{}
		\\ [2.0cm]
		\HRule{1.5pt} \\
		\LARGE \textbf{\uppercase{Template Title}
		\HRule{2.0pt} \\ [0.6cm] \LARGE{Subtitle} \vspace*{10\baselineskip}}
		}
\date{}
\author{\textbf{Author} \\ 
		Who? \\
		Where? \\
		When?}

\maketitle
\newpage

\tableofcontents
\newpage

% ------------------------------------------------------------------------------

\section{高等代数}

%\begin{theorem}
%    This is a theorem.
%\end{theorem}

%\begin{proposition}
%    This is a proposition.
%\end{proposition}

%\begin{principle}
%    This is a principle.
%\end{principle}

% Maybe I need to add one more part: Examples.
% Set style and colour later.

\subsection{线性变换}
                 
\begin{theorem}
    设$V_1,V_2,\ldots ,V_s$是线性空间V的s个非平凡子空间,则V中至少有一个向量不属于$V_1,V_2,\ldots ,V_s$中任何一个。
\end{theorem}
\noindent 不属于其中一个是“并”的意思,因此数学归纳法时,不能单纯的用空间的和来表示不属于任何一个。\\
证明对于不同的数$k_1,k_2$,向量$k_1\alpha +\beta ,k_2\alpha +\beta $不属于同一个$V_i,(1\leq i\leq s-1)$
\begin{theorem}
    如果$T_1,T_2,\ldots ,T_m$是非零线性空间V的m个两两不同的线性变换,则在V中必存在向量$\alpha $,使$T_1\alpha ,T_2\alpha ,\ldots,T_m\alpha $也两两不同。
\end{theorem}
\noindent 每两个线性变换均满足$\exists \alpha,T_i\alpha \neq T_j\alpha$,从而每个$V_{ij}\neq V$\\
(在V空间中,$V_{ij}=\{x\mid x\in V,T_i \alpha=T_j \alpha\}$,即$V_{ij}$是V的子空间)\\
  由以上可知,$V_{ij}$是V的非平凡子空间或V是零空间。
因此至少有一个向量不属于$V_{ij}$中任何一个,即所有$T_i\alpha \neq T_j\alpha$,即$T_1\alpha ,T_2\alpha ,\ldots,T_m\alpha $两两不同。

\begin{proposition}
    在V中,$V_{ij}=\{x\mid x\in V,T_i \alpha=T_j \alpha\}$,即$V_{ij}$是V的子空间。\\
    设T是数域F上n维线性空间V的一个线性变换,任取向量$\alpha\in V$,则有$T^{-1}\alpha\in V.$             
\end{proposition}

\subsection{分块矩阵}

\noindent 公式一:$\begin{vmatrix}
    A&B \\
    0&D
  \end{vmatrix}=(detA)(detD)$\\
公式二:$\begin{vmatrix}
    A&B \\
    C&D
  \end{vmatrix}=(detD)(det(A-BD^{-1}C))$\\
公式三:设A,B,C,D是$n\times n$阶矩阵,若A,B,C,D其中之一是零矩阵,则
$$\begin{vmatrix}
    A&B \\
    C&D
  \end{vmatrix}=(detAD-BC)$$
\begin{theorem}
    A,B分别是$n \times m$和$m \times n$矩阵,则
    $$\begin{vmatrix}
        E_m&B \\
        A&E_n
      \end{vmatrix}=\left\lvert E_n-AB\right\rvert=\left\lvert E_m-BA\right\rvert  $$
      若$\lambda \neq 0$,则$\left\lvert \lambda E_n-AB\right\rvert =\lambda^{n-m}\left\lvert \lambda E_m-BA\right\rvert $
\end{theorem}
以上表示,若AB为任意两个n阶方阵,则AB与BA有相同的特征多项式,特征值相同。\\
AB=BA时,有以下性质:
\begin{enumerate} 
    %此为有序列表,无序列表为itemize
        \item (A+B)的n次方可以直接二项式展开; 
        \item 有部分题交换后可以直接凑出特殊矩阵;
        \item \textbf{A,B有至少有一个相同的特征向量,但其他的特征向量不一定相同},这个性质给同时对角化,同时上三角化提供了条件,另一方面给有限个矩阵可交换的命题进行归纳时提供了很好的条件. 
    \end{enumerate}

\subsection{矩阵乘积AB的性质}
\begin{enumerate} 
    \item 特征多项式:AB与BA的非零特征值的几何,代数重数都相等$\Leftrightarrow$ AB可对角化时,BA也可对角化
    \item 特征值:若A和B都是正定矩阵,那么AB的特征值是正数。特别地,如果$B=A^T$,那么AB的特征值就是奇异值的平方
    \item 交换性与多项式表示:AB=BA,A可对角化,则A,B可以同时对角化,而且B可以写成A的多形式形式。
    \item 秩的性质:$r(AB)+n\geq r(A)+(B)$
    \item 映射方面的解释:$r(AB)=dim Im(AB)=dim ImA\mid ImB$
    \item 可对角化:AB可对角化$\Leftrightarrow$ BA,A可对角化,则A,B可以同时对角化,而且B可以写成A的多形式形式。\\
    $\begin{pmatrix}
        I&A \\
        B&I
      \end{pmatrix}$经过两种形式的变换以后可以化成 $\begin{pmatrix}
        I-AB&O \\
        O&I
      \end{pmatrix}$以及$\begin{pmatrix}
        I&O \\
        O&I-BA
      \end{pmatrix}$这两种形式;注意到这两个矩阵之间是相似的关系$\Rightarrow $AB和BA的特征值的代数重数和几何重数分别相等
    \item $A\in P^{m \times n},B\in P^{n \times m}$,证明:$r(k(I_m-AB)^k)+n=r(k(I_n-BA)^k)+m,\forall k\in N^+$  \\
    对于k=1而言,当然是很常见的结论,但是对于更大的k怎样考虑呢?首先看到矩阵的幂次很自然地想到了相似(因为引入相似的概念可以很好地解决矩阵的幂次问题)\\
    也就是说 $\begin{pmatrix}
        I-AB&O \\
        O&I
      \end{pmatrix}$以及$\begin{pmatrix}
        I&O \\
        O&I-BA
      \end{pmatrix}$这两种形式相似,所以考虑一下Jordan标准型就出来了。
\end{enumerate}



\newpage

% ------------------------------------------------------------------------------
% Reference and Cited Works
% ------------------------------------------------------------------------------

\bibliographystyle{IEEEtran}
\bibliography{References.bib}

% ------------------------------------------------------------------------------

\end{document}
